
\section{Еще про контрольные переменные}

\begin{frame}{Повторение}

Мы узнали, что контроль нужен, чтобы
\begin{itemize}[<+->]
    \item Снизить дисперсию. Почему так происходит?
    \item Потому что \textbf{ковариаты} <<съедают>> часть дисперсии
\end{itemize}

\pause
Что еще:

\begin{itemize}
    \item Получить \textbf{гетерогенные эффекты}. \textbf{Предикторы}
    \item Попробовать проанализировать неэкспериментальные данные \textbf{Смесители (confounders)}
    \item Разобраться в эффекте \textbf{Медиаторы}
    \item Испортить несмещенность \textbf{Bad Control}
\end{itemize}

Сегодня мы обо всем этом коротко поговорим, про большую часть эффектов поговорим позже
    
\end{frame}

\begin{frame}{Получить гетерогенные эффекты}
\begin{itemize}
    \item Помните исследование про переселение в Чикаго?
\end{itemize}
\end{frame}

\subsection{Плохой контроль}

\begin{frame}{Плохой контроль\footnote{\cite[Глава 3.2.3]{angrist2008mostly}}}

Примеры:
\begin{itemize}
    \item Контроль на место работы при исследовании влияния образования на доходы
\end{itemize}

схема

Интуиция:
\begin{itemize}
    \item Люди, которые получили white collar job без образования сами по себе крутые -- sample bias
\end{itemize}

\end{frame}

\begin{frame}{Плохой контроль}

Примеры:
\begin{itemize}
    \item Контролировать на явку в эксперименте с выборами
    \item Контроль на место работы при исследовании влияния образования на доходы
\end{itemize}
    
\pause
    
Формально:
$$(Y_1, Y_0, X) \perp T\text{ --- не выполнено}$$
$$(Y_1, Y_0, X_1, X_0) \perp T\text{ --- предположим это}$$

\begin{gather*}
E(Y|X=1 T=1) - E(Y|X=1 T=0) = \\
E(Y_1|X_1=1) - E(Y_0|X_0=1 T=0) = \\
E(Y_1|X_1=1) - E(Y_0|X_0=1) =
\\E(Y_1 - Y_0|X_1=1) + (E(Y_0|X_1=1) - E(Y_0|X_0=1))= \\
\text{LATE} + \text{Смещение выборки}
\end{gather*}

% про proxy тут???
% good cop
% это не то
% Mediation? 

% NEED A TABLE SLIDE
% NEED A more statistical foundation here. WHAT IS!
% need a better inference

\end{frame}

\subsection{Эффект посредничества}

% \begin{frame}{Mediation effect}
% Пример: Контролировать на явку в эксперименте с выборами

% схема

% \end{frame}