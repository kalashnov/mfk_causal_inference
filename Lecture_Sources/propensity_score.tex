
\section{Propensity score}

\begin{frame}{Balancing score}
\begin{itemize}
    \item Достаточная статистика
    $$T_i \perp (Y(1)_i, Y(0)_i | X_i) <=> T_i \perp (Y(1)_i, Y(0)_i)  | e(X_i)$$
    \item Propensity score:
    $e(X_i)=P(T_i = 1 | X_i)$
    \item Смысл леммы: чтобы избавиться от смещения в оценке $\tau$, вместо всех ковариат достаточно проконтролировать на  меру склонности
    \end{itemize}
\end{frame}

%\begin{frame}{Propensity score и balancing property}
%\begin{itemize}
 %   \item Propensity score:
  %  $e(X_i)=P(T_i = 1 | X_i)$
   % \item Доказательство на доске (from Imbens, Rubin, chapter 12.3)
%    \item Смысл леммы: чтобы избавиться от смещения в оценке $\tau$, вместо всех ковариат достаточно проконтролировать на  меру склонности
%\end{itemize}

%\end{frame}

\begin{frame}{Способы проконтролировать на propensity score}
\begin{itemize}
    \item blocking
    \item matching
    \item weighting
\end{itemize}
Это все просто про перевзвешивание
\end{frame}

%\begin{frame}{Не только ATE}

%\begin{itemize}
 %   \item Cредний эффект воздействия (\textbf{average treatment effect}):
%$$
%\text{ATE} = \mathbb{E}(\tau) =  \mathbb{E}(Y_1 - Y_0)
%$$

%зачем? почему отличается?
%\item Cредний эффект воздействия на задействованных (\textbf{average treatment on the treated}):
%$$
%\text{ATT} = \mathbb{E}(\tau|T=1) = \mathbb{E}(Y_1 - Y_0|T=1)
%$$
%\item Cредний эффект воздействия на незадействованных (\textbf{average treatment on the non-treated}):
%$$
%\text{ATnT} = \mathbb{E}(\tau|T=0) =  \mathbb{E}(Y_1 - Y_0|T=0)
%$$
%\end{itemize} 
    
%\end{frame}



%\begin{frame}{Бонус: propensity score, снижижающий дисперсию}
 %   Устно
%\end{frame}


%\begin{frame}{Double-robustness}
%\begin{itemize}
 %   СДЕЛАТЬ ЗДЕСЬ ПОЛУЧШЕ ССЫЛКУ НА CHERNOZHUKOV
  %  \item Давайте освежим. Какие 2 способа есть, чтобы избавиться от проблемы Confounders?
   % \item Когда мы получим несмещенную оценку?
    %\item А давайте сделаем и то и другое и получим Double robustness
%\end{itemize}
%\end{frame}
