
\section{Propensity score}

% \begin{frame}{Balancing score}
% очень не хватает иллюстрации проблемы
% RUBIN IMBENS 2006
% \end{frame}

% \begin{frame}{Balancing score}
% очень не хватает иллюстрации идеи
% \end{frame}


\begin{frame}{Balancing score\footcite[Можно почитать у][]{enikolopov2009ecm}\footcitetext[Раздел 3.3]{angrist2008mostly}}
\begin{itemize}
    \item Достаточная статистика
    $$T_i \perp (Y(1)_i, Y(0)_i)| X_i <=> T_i \perp (Y(1)_i, Y(0)_i)  | e(X_i)$$
    \item Propensity score:
    $e(X_i)=P(T_i = 1 | X_i)$
    \item Смысл леммы: чтобы избавиться от смещения в оценке $\tau$, вместо всех ковариат достаточно проконтролировать на  меру склонности. Доказательство у \textcite[Глава 15]{imbens2015causal} и \textcite{rubin1978bayesian}
    \end{itemize}
\end{frame}

\begin{frame}{Способы применить propensity score}
\begin{itemize}
    \item Blocking
    \item Matching
    \item Weighting
\end{itemize}
\end{frame}

\begin{frame}{Blocking}
\begin{itemize}
    \item Вычисляем propensity score.
    \item Разбиваем наблюдения по блокам: 0.2-0.4, 0.4-0.6, 0.6-0.8
    \item $\text{ATE} = \frac{N_H}{N} \left(\frac{1}{N_{TH}}\sum_{T=1, S=H} Y - \frac{1}{N_{CH}}\sum_{T=0,S=H} Y \right)+ 
\frac{N_L}{N} \left(\frac{1}{N_{TL}}\sum_{T=1, S=L} Y - \frac{1}{N_{CL}}\sum_{T=0,S=L} Y \right)$
\end{itemize}

\begin{itemize}
    \item Что плохого в пропуске данных?
\end{itemize}
\end{frame}

\begin{frame}{Matching}
\begin{itemize}
    \item Вычисляем propensity score.
    \item Находим наблюдения с самыми близкими значениями propensity score. Остальные выбрасываем
    \item Вычисляем обычный ATE
\end{itemize}
\end{frame}

\begin{frame}{Weighting}
\begin{itemize}
    \item Вычисляем propensity score.
    \item Берем наблюдения из диапазона 10-90
    \item $\text{ATE} = \frac{1}{N}\sum_{T=1} \frac{1}{e(X)} Y - \frac{1}{N}\sum_{T=0} \frac{1}{1 - e(X)} Y$
\end{itemize}
    \item Почему бы все не взять?
\end{frame}

% про оценку плотности тут!
% weighting тоже

\begin{frame}{Все это и есть взвешивание}
\begin{itemize}
    \item Matching -- веса 0/1
    \item Blocking: $\frac{N_H}{N_{TH}}$ и $\frac{N_L}{N_{CH}}$
    \item Weighting: $\frac{1}{e(X)}$ и $\frac{1}{1 - e(X)}$
\end{itemize}


Как получить ATT?
\begin{itemize}
    \item Blocking: $\frac{N_{TH}}{N_{T}}$ и $\frac{N_TL}{N_{T}}$
    \item Weighting: 1, $\frac{e(X)}{1 - e(X)}$
\end{itemize}
\end{frame}




% \begin{frame}{На семинаре}
% \begin{itemize}
%     \item Matching -- веса 0/1
%     \item Blocking: $\frac{N_H}{N_{TH}}$ и $\frac{N_L}{N_{CH}}$
%     \item Weighting: $\frac{1}{e(X)}$ и $\frac{1}{1 - e(X)}$
% \end{itemize}


% Как получить ATT?
% \begin{itemize}
%     \item Blocking: $\frac{N_{TH}}{N_{T}}$ и $\frac{N_TL}{N_{T}}$
%     \item Weighting: 1, $\frac{e(X)}{1 - e(X)}$
% \end{itemize}
% \end{frame}



% \begin{frame}{Balancing score}
% эквивалентность и предел по блокам
% \end{frame}

% \begin{frame}{Balancing score}
% Почему это все про нелинейность
% \end{frame}

% \begin{frame}{Double robustness}
    
% \end{frame}


% \begin{frame}{Balancing score}
% очень не хватает вопросов дисперсии
% \end{frame}

% \begin{frame}{Balancing score}
% Больше про моделирование этой штуки. Как получить?
% \end{frame}




% В другую лекцию: про overlap

% \begin{frame}{Balancing score}
% Больше про моделирование этой штуки. Как получить?
% \end{frame}

% horowitz tompson!